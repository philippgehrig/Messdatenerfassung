\chapter{Versuch 1: Kapazitätsmessung eines unbekannten Kondensators}

\section{Vorbereitung}
\subsection*{Benötigte Geräte}

\begin{tabular}[h]{c|c}
    Kondensators unbekannter Größe & unbekannt \\
    \hline
    Widerstand 4,7 k$\Omega$& \\
    \hline
    Funktionsgenerator & T3AFG80\\
    \hline
    Digital-Multimeter & Fluke TRUE RNS MULTIMETER\\
    \hline
    Oszilloskop & Keysight DSOX1102A
    \label{tab:Versuch 1: Geräte}
\end{tabular}

\subsection*{Ziel des Versuchs}
Für einen unbekannten Kondensator ist ein Bereich 1nF <= C\textsubscript{x} <= 10nF 
angegeben. Ziel dieses versuches ist es, die kapazität des Kondensators genauer
zu bestimmen und eine Fehlerrechnung durchzuführen.

\section{Versuchsdurchführung}

Zunächst wurde der im Datenblatt angegebene Wiederstand von 4,7 k$\Omega$ 
mit dem Digital-Multimeter überprüft. Dieser lag mit 4,583 k$\Omega$ im 
Toleranzbereich von 5\%. Für den Kondensator war BlackBox 27.

Zur Brrechnung des wird die Formel für die Kapazität eines Kondensators verwendet:
\begin{equation}
    Q(t) = C \cdot V \left(1 - e^{-\frac{t}{RC}}\right)
\end{equation}

XXXXXX Abbildung Funktion XXXXXX

\subsection*{Versuchsaufbau}

XXXXXX Abbildung Versuchsaufbau XXXXXX

\subsection*{Bestimmung der Kapazität}

\section{Fehlerrechnung}



