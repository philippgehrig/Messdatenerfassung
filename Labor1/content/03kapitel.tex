\chapter{Versuch 2}

\section{Benötigte Geräte}

Für dieses Experiment benötigen wir die Folgenden Geräte:

\begin{tabular}[h]{c|c|c}
    Gerät & Anzahl & Produktbeeichnung\\
    \hline
    Oszilloskop & 1  & Keysight DSOX1102A\\
    \hline
    % Frequenzgenerator & 1 & TELEDYNE T3AFG80\\
	% \hline 
	Digital-Multimeter & 1 & Fluke TRUE RMS MULTIMETER\\
	\hline 
	Widerstand 1k & 9 &  \\
	\hline 
	Widerstand 10k & 1 &  \\
	\hline
	LED & 8 & \\
	\hline
	DAC & 1 &  \\
	\hline
	Kondesator 0,1 $\mu$F & 2 & \\
	\hline
	Kondesator 10 $\mu$F & 1 & \\
	\hline
	Kondesator 150 pF & 1 & 
        \label{tab:Materialliste Versuch 1}
\end{tabular}

\section{Änderungen im Vergleich zu Versuch 1}

Zunächst einmal wird nun nur noch eine Referenzspannung von 2,56V 
(anstatt 5,12V)

\section{Audbau der Schaltung}
\section{Monotonie und Nichtlinearität}
\section{Einschwingverhalten}